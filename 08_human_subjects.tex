
{\noindent \bf \Large PROTECTION OF HUMAN SUBJECTS}
\vspace{-1em}
% Refer to Part II, Supplemental Instructions for Preparing the Human Subjects Section of the Research Plan.

% This section is required for applicants answering "yes" to the question "Are human subjects involved?" on the R\&R Other Project Information form. If the answer is "No" to the question but the proposed research involves human specimens and/or data from subjects applicants must provide a justification in this section for the claim that no human subjects are involved.

% Do not use the protection of human subjects section to circumvent the page limits of the Research Strategy.

\subsection*{Risks to Human Subjects}\vspace{-1em}
Human subjects will comprise of physicians (including trainees) and will be involved in three types of studies:
\one Survey studies to conduct background research, which is necessary to establish a comprehensive and statistics-driven understanding of physicians' work;
\two Interview studies to obtain an in-depth understanding physicians' work and elicit requirements for an AI-enabled system, which is necessary for informing and guiding the system design;
\three Lab studies where physicians will interact with a software prototype of the proposed systems using a desktop computer, which is necessary for evaluating the performance and usability of the systems.

We expect to conduct surveys/questionnaires (100-150 subjects each time), interviews (5-10 subjects each time), participatory design sessions (5-10 subjects each time) and usability evaluations (10-20 subjects each time).

There are no specific age or health condition requirements for the human subjects. There will be no involvement of vulnerable populations.

Specific inclusion criteria will be based on the data type processed by AI. For example, to study an AI-enabled system for processing histological data, we will include pathologists. There are no exclusion criteria.

A survey, involving answering a questionnaire, takes 15-30 min; an interview or lab study takes about 1 hour.

For survey studies, quantitative data will be collected from subjects' response to the questionnaire; interview and lab studies will be audio and video recorded; in lab studies, we will also record the computer screens and log user input data (\eg mouse and keyboard). Collected data will be encrypted and stored on local servers on campus and will be backed up periodically.

Subjects will be identified anonymously (e.g., as `P1, P2, ...'). No personally identifiable information will be collected or retained.

The specific tasks involve answering a questionnaire or interview questions and using a computer software, which incur minimum risks as these are activities similar to how the subjects (physicians) communicate with others and use computers in their daily work.

\subsection*{Adequacy of Protection Against Risks}\vspace{-1em}
Subjects will be recruited primarily via online community, words-of-mouth and internal mailing lists.
Informed consent will be obtained as per UCLA IRB regulation prior to each study.
The purpose, procedure, tasks and minimum risks of the study will be described as part of the consent form.

\subsection*{Potential Benefits of Research to Human Subjects and Others }\vspace{-1em}
The topics of survey and interview questions as well as the software prototypes in the lab study are germane to the medical profession of the subjects. Thus the subjects who participate in our studies will have the benefits of learning about the general knowledge of AI and how AI can be applied in medicine, which is knowledge that complements and/or advances the subjects' own medical expertise. 

The research supported by these studies are important to physicians' work, as such research leads to a new generation of clinical decision support systems that in the long run will mitigate physicians' workload and use AI to assist and support physicians' work.


\subsection*{Importance of Knowledge to be Gained}\vspace{-1em}
Since the risk is minimum, it is thus significantly outweighed by the benefits that can be brought forth by the studies.